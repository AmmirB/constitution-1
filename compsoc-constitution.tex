% Preamble
% ---
\documentclass{article}

% Packages
% ---

\usepackage{geometry}

% Commands
% --

\newcommand*{\SignatureAndDate}[1]{%
    \par\noindent\makebox[2.5in]{\hrulefill} \hfill\makebox[2.0in]{\hrulefill}%
    \par\noindent\makebox[2.5in][l]{#1}      \hfill\makebox[2.0in][l]{Date}%
}%

\geometry{
  a4paper,
  total={210mm,297mm},
  left=20mm,
  right=20mm,
  top=20mm,
  bottom=20mm,
}

% Document
% ---

\begin{document}

\title{CompSoc Constitution}
\author{The University of Edinburgh Computing \& Artificial Intelligence Society}
\date{April 20, 2012}
\maketitle

\section{General}

The name of the Society shall be ``CompSoc'' or ``The University of
Edinburgh Computing and Artificial Intelligence Society''.

\begin{enumerate}
  \item The aims of the Society shall be as follows:-
    \begin{itemize}
      \item To provide a forum for members to discuss issues relating to computer science and computing in general;
      \item To facilitate social interaction amongst people with a common interest in computer-related issues;
      \item To provide members with assistance with computer-related problems, including support for study;
      \item To promote liaison between the local academic and business communities, with regard to fostering future employment opportunities;
      \item Share experience and knowledge of work in Informatics by:
        \begin{itemize}
          \item Holding special interest group meetings to provide members with a platform for discussion and an opportunity to learn new skills.
          \item Holding informal seminars with workers and researchers in Informatics from industry or other academic institutions.
          \item Allowing members to have access to various Informatics-related programming tools in order to gain experience in using them to solve related problems.
        \end{itemize}
    \end{itemize}


  \item The benefits of the Society are:
    \begin{itemize}
      \item Free Pizza
      \item Allow likeminded individuals the chance to socialise and network
      \item Allow the indulgence of all geekery
      \item Organised events to benefit those interested in technology
    \end{itemize}
    

  \item The society shall abide by any applicable laws, bylaws and guidelines of the Edinburgh University Students’ Association in relation to recognised societies

  \item Membership shall be open to all matriculated students of Edinburgh University.

  \item Non-students may be members of CompSoc, vote in general meetings and stand for committee-positions other than President, Secretary and Treasurer

  \item Membership shall be at least 75\% matriculated students of Edinburgh University.

  \item All members who are not matriculated students of The University of Edinburgh shall pay at least twice the annual subscription.

  \item The society’s cheques must require two signatories; One must be the Treasurer and the other President or Secretary.

  \item The society has taken and will continue to take all necessary steps to ensure that our meetings, events and socials are accessible to all, irrespective of any disability. 

  \item The society has ensured and will continue to ensure that it complies with any relevant data protection legislation.

  \item EUSA considers the ruling society constitution to be that which is displayed on the Society Profile

  \item Re-registration of the society must be submitted prior to re-registration deadline set by EUSA

  \item The society believes that discrimination or harassment, direct or indirect,
    based on a person’s gender, age (except where it relates to licensing
    laws), race, skin colour, nationality, religious belief, socioeconomic background,
    disability, HIV status, sexual orientation, gender reassignment,
    family situation, domestic responsibilities or any other irrelevant distinction,
    is detrimental to the society, the University and wider society, and
    will not be tolerated.
    
  \item The society shall remain ``Most Excellent''

\end{enumerate}
% End section 'General'



\newpage{}
\section{The Committee}

\begin {enumerate}
  
  \item All office-bearers shall be subject to election annually.

  \item The President, Secretary and Treasurer of the society shall be matriculated students of The University of Edinburgh.

    \begin{itemize}
      \item The President shall be ultimately responsible for the conduct of the society
      \item The Secretary shall be responsible to the President for the administration of the society
      \item The Treasurer shall be responsible to the President for the finances of the society.
    \end{itemize}
    
  \item Society office bearers will attend annual society training as outlined by the Societies Team
  
  \item Other elected positions on the committee may be Vice President, Social
    Secretary, Technical Secretary, Second Year Representative, 
    Third Year Representative and Fourth Year Representative.

  \item The year representatives must be students which will be enrolled in that year the semester 
    following the AGM. At the time of the AGM the Second Year Representative should be a first year, etc.

  \item The following are guidelines as to what role the person in these positions shall have in the society:
    \begin{enumerate}
      \item The Vice President shall oversee and work with the SIGs as well as
        being responsible for the STMUs (Finding speakers and handling venues).
      \item The committee may find volunteers responsible for promotional materials
        being made for events. The Secretary shall be ultimately
        responsible for news and updates being posted to the website and
        maintaining any social networks CompSoc may have a presence on,
        although this may be delegated to others if the Secretary feels this
        would be beneficial.
      \item The Secretary is also ultimately responsible for the creation of any
        society apparel such as hoodies, although this may be delegated to others
        if the Secretary feels this would be beneficial. The Social Secretary
        shall be responsible for booking venues (Other than STMU related venues 
        and Hackathon related venues) for and running the societies
        offcial events.
      \item The Year Representatives shall be responsible for communicating to
        and promoting the society to students in their year.
    \end{enumerate}

    It is important to note that the above are suggestions to what those elected
    to the positions should be responsible, not absolute rules. Delegation is
    encouraged and necessary, but those in the relevant positions should take
    responsibility for delegating the tasks and making sure they get done.
    
    \item Non elected positions on the committee are as follows:-
      \begin{enumerate}
        \item To ensure cooperation between the School of Informatics and CompSoc,
          the school-representatives for the school of informatics automatically
          have a place and vote on the committee. It is entirely up to
          the representative to what degree he or she wants to participate in
          the committee.
        \item The leaders of the Special Interest Groups have a place and vote on
          the committee (see below, ”Special Interest Groups”)
        \item A First Year Representative shall be taken onto the committee by the
          end of October of the first semester of every term. It is up to the
          committee to select a first year representative, but the commitee is
          encouraged to facilitate a vote by first year students.
      \end{enumerate}

    \item The committee may find it helpful to appoint members to additional positions
      on the committee after the elections for handling things like the CompSoc website.
      These must be co-opted onto the committee by a two third majority of the current committee.

\end{enumerate}
% End section 'The Committee'


\newpage{}
\section{Special Interest Groups (SIGs)}

\begin{enumerate}

  \item Any member of CompSoc may, and is encouraged to, start a special interest
    group pertaining to any common interest the members may have.
    What follows are guidelines on the rights and rules regarding SIGs.

  \item When forming a new SIG a request should be handed in to the committee
    outlining the main goals and purpose of the SIG.

  \item The following points will be considered by the committee when processing
    the application of a new SIG:-
    \begin{enumerate}
      \item The SIG should have a clearly defined goal, purpose or other reason
      for existence deemed appropriate by the committee.
      \item The SIG should have at least 4 members. In extraordinary circumstances
      this can be overruled by the vote of 2/3 of the committee.
      \item The SIG should have current CompSoc-member as a leader. The
      leader shall be responsible for running the group and reporting back
      to the Vice President and the CompSoc committee. It is encouraged
      that the selection of a leader should be done democratically within the
      group.
    \end{enumerate}

  \item New SIGs must be approved by two thirds of the committee.

  \item Once an SIG is accepted:-
    \begin{enumerate}
      \item The leader of any accepted SIG will automatically have a seat and a
        vote on the CompSoc committee.
      \item is entirely up to the SIG how it wants to handle signing up members,
        arranging meetings or other matters pertaining to the general
        administration of the SIG.
      \item The SIG has the right to receive support from the committee within
        reason, including but not limited to financial support, creation of
        advertising materials and free promotion on the CompSoc website,
        newsletter and Facebook-group. Requests for such support should
        be given in writing to the committee.
      \item The leader of an SIG should continuously keep the committee informed
        of the activities of the SIG.
    \end{enumerate}

    
  \item If the committee feels an SIG’s activities has become incompatible with
    the interests and aims of the society, the committee may, by a two third
    majority, decide to dis-associate CompSoc from the SIG.

  \subsection{Leadership of SIGs}

  \item The choice of a leader should be entirely up the SIG. If a major dispute
    about leadership of an SIG is brought to the attention of the committee, a
    member of the committee will temporarily act as a leader for the SIG and
    attempt to find a solution. If no agreement is made within a reasonable
    amount of time the committee may choose to appoint a leader, or decide
    to dis-credit the SIG.

  \item When an SIG gets a new leader the change in leadership should be reported
    to the committee by the old leader.

  \subsection{Special Cases}

  \item Note that the above are guidelines which should be followed in the general
    case. In some instances it will be impractical for an SIG to follow the
    above rules. This may include external groups or clubs with a university
    division that wishes to associate itself with CompSoc or on-campus groups
    associated with larger tech-companies. In these cases it will be entirely
    at the committee’s discretion how the relationship between the SIG and
    CompSoc will work. The SIG will still require a two third majority of the
    committee to be approved.

\end{enumerate}
% End section 'Special Interest Groups'





\newpage{}
\section{AGM}

\begin{enumerate}

  \item Before the AGM the committee shall provide a document which outlines
    the activities and events which took place since the last AGM. This document
    should give members attending the AGM an insight into the activities
    of the society and give the new committee a guide to what they
    should go on to do in their term.

  \item All members shall be entitled to stand and to vote in elections

  \item All members must receive at least 14 days written notification of the annual
    general meeting and of elections not held at the AGM.

  \item The society shall also inform the EUSA Societies’ Office of the date, time and
    place of the society’s AGM.

  \item The AGM of the society must take place between week 7 and week 11 of
    semester 2.

  \item The quorum of a general meeting shall be 10\% of the current membership
    unless:-
    \begin{enumerate}
      \item The number of members is less than 50, in which case the quorum
        shall be 5 members
      \item The number of members exceeds 200 in which case the quorum shall
        be 20 members.
    \end{enumerate}

  \item Votes shall be cast by the raising of hands, unless any member requests
    otherwise, in which case votes shall be cast by secret ballot.

  \item Current Office Bearers must be assigned to the Society Profile immediately
    upon election

  \item The following shall be the procedure for amending the constitution:-
    \begin{enumerate}
      \item The committee must make the amended constitution available to
        society members no later than 14 days prior to a general meeting.
      \item Suggested changes to an amended constitution must be submitted to
        the committee no later than two days prior to the general meeting.
      \item Changes to an amended constitution will be presented at the general
        meeting and will be voted upon. Changes to an amended constitution
        must be passed by a two third majority. After all changes have been
        voted on, the amended constitution with the passed changes will be
        voted on.
      \item The amended constitution must be passed with a two third majority
        at the general meeting.
    \end{enumerate}
    
  \item When elections are held there will always be an option to re-open nominations.
    This option will be presented as an additional candidate, “Ron”.
    If Ron gets the greatest amount of votes the position will be kept open
    and an EGM shall be called by the committee to fill the position. There
    will not be a “Ron” candidate for the positions of President, Secretary,
    Treasurer. Nominations may be made at the AGM, subject to the nominee's
    consent.

  \item If no nominations are received for any of President, Secretary or Treasurer,
    the incumbent committee shall call an EGM within 4 weeks. If all 3
    positions are not filled following an EGM, the incumbent committee shall
    inform EUSA that the society has become defunct. Anybody can reform
    the society by contacting EUSA, however this must be done within 3 years
    or all society assets will be lost.

\end{enumerate}
% End section 'AGM'





\newpage{}
\section{EGM}

\begin{enumerate}

  \item An EGM can be called in the following ways:-
    \begin {enumerate}
      \item The resignation of any elected committee-member will trigger an
        EGM at the earliest opportunity unless an AGM is held within 4
        weeks.
      \item A quorate committee may vote, by simple majority, to hold an EGM
        on constitutional amendments.
      \item A quorate committee may vote, by two thirds majority, to hold an
        EGM to re-elect a committee member. This member would be entitled
        to run again for the same position and if re-elected could not be
        removed from the position by another EGM until an AGM is held.
      \item Any member of the society may bring about an EGM to amend the
        constitution or replace an elected member/members of the committee
        by collecting signatures of at least a third of the society, with a
        minimum of 30 members.
    \end{enumerate}

  \item The committee shall decide on a date for an EGM which must be within
    4 weeks of it being called for. If an EGM is called for outside of term-time
    the committee must decide on a date within 4 weeks of the start of term.

  \item All members must receive at least 14 days written notification of an EGM.
    An e-mail suffices as a written notification.

  \item An EGM can be called with the purpose of either amending the constitution
    or re-electing a committee member.
  
  \item The quorum of a general meeting shall be the same as that for an AGM.

  \item Votes shall be cast by the raising of hands, unless any member requests
    otherwise, in which case votes shall be cast by secret ballot.

\end{enumerate}
% End section 'EGM'





\newpage{}
\section{Committee Meetings}

\begin{enumerate}

  \item The committee shall meet at least once every 14 days in term-time.
  
  \item The quorum of a committee meeting shall be 75\% of the members of the
    committee with at least two of the president, treasurer and secretary.
  
  \item As a general rule the meetings shall also be open to any members who wish
    to discuss an item on the agenda with the committee or raise any other
    issue. However, if for any reason any member of the committee wants a
    meeting to be restricted to the committee only, a request should be made
    to the Secretary and the Secretary in concurrence with the President can
    restrict the meeting to committee members only

  \item The minutes of a committee meeting must be made available to members
    of the society within a week of the meeting being held.

\end{enumerate}
% End section 'Committee Meetings'




\section{Signatures}

\begin{enumerate}
  \item The acting executive committee shall sign one or more paper copies of the constitution
    to show agreement with the above terms:
\end{enumerate}

% Signatures
% ---
\vspace{2cm}
\SignatureAndDate{President}
\vspace{2cm}
\SignatureAndDate{Secretary}
\vspace{2cm}
\SignatureAndDate{Treasurer}


\end{document}
